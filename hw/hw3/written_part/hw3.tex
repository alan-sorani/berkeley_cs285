\documentclass[10pt]{article}

%---GEOMETRY

\usepackage{geometry}
   \geometry{verbose,tmargin=0.75in,bmargin=0.75in,%
     lmargin=0.7in,rmargin=0.4in,headheight=0.25in,%
     headsep=0.2in,footskip=0.4in}%

%---COMMENTS---
\usepackage{comment}

%---CODE OUTPUT---
\usepackage{listings}
\lstset
{ %Formatting for code in appendix
    language=Matlab,
    basicstyle=\footnotesize,
    numbers=left,
    stepnumber=1,
    showstringspaces=false,
    tabsize=4,
    breaklines=true,
    breakatwhitespace=false,
    escapeinside={(*}{*)}
}

\newcommand{\codeword}[1]{\texttt{#1}}

%---HYPERLINKS---

\RequirePackage{hyperref}
\hypersetup{
    colorlinks=true, %set true if you want colored links
    linktoc=all     %set to all if you want both sections and subsections linked
}

\begin{comment}
    citecolor=stycitecolor,
    filecolor=styfilecolor,
    linkcolor=stylinkcolor,
    urlcolor=styurlcolor
\end{comment}

%---MATHS PACKAGES---

\usepackage{math}

%---THEOREMS

\usepackage{altthms}

\theoremstyle{definition}

\renewtheorem{exercise}{Exercise}[section]
\renewcommand*{\theexercise}{\thesection-\arabic{exercise}}

%---INDEX---

\usepackage{imakeidx}
\makeindex

%---PLOTS---
\usepackage{graphicx}
\graphicspath{ {./images/} }

%---BIBLIOGRAPHY---

\usepackage[backend=biber, style=alphabetic]{biblatex}
\addbibresource{bibliography.bib}

%---ENUMERATION---

\usepackage{enumitem}

%---TITLE---

\title{CS285: Deep Reinforcement Learning \\ Assignment 3 \\ Written Report}
\author{Alan Sorani}
\date{\today}

\begin{document}

\maketitle

\section{Multistep Q-Learning}

\subsection{TD-Learning Bias}

Assume that $\hat{Q}$ is a noisy unbiased estimate for $Q$. Then the Bellman backup $\mathcal{B}\hat{Q} \coloneqq r\prs{s,a} + \gamma \max_{a'} \hat{Q}\prs{s', a'}$ is a \textbf{biased estimate} of $\mathcal{B}Q$. We have
\begin{align*}
\mathbb{E}_{\tau \sim p_\theta}\brs{\mathcal{B}\hat{Q}}
&=
\mathbb{E}_{\tau \sim p_\theta}\brs{r\prs{s,a} + \gamma \max_{a'} \hat{Q}\prs{s', a'}}
\\&=
\mathbb{E}_{\tau \sim p_\theta}\brs{r\prs{s,a}} + \gamma \mathbb{E}_{\tau \sim p_\theta}\brs{\max_{a'} \hat{Q}\prs{s', a'}} \text{,}
\end{align*}
and similarly
\begin{align*}
\mathbb{E}_{\tau \sim p_\theta}\brs{\mathcal{B}{Q}}
&=
\mathbb{E}_{\tau \sim p_\theta}\brs{r\prs{s,a}} + \gamma \mathbb{E}_{\tau \sim p_\theta}\brs{\max_{a'} Q\prs{s', a'}} \text{,}
\end{align*}
so $\mathcal{B}\hat{Q}$ is an unbiased estimate of $\mathcal{B}Q$ if and only if
\[\mathbb{E}_{\tau \sim p_\theta}\brs{\max_{a'} \hat{Q}\prs{s', a'}} = \mathbb{E}_{\tau \sim p_\theta}\brs{\max_{a'} Q\prs{s', a'}} \text{.}\]

This is not true. Consider an MDP with the action space $\mathcal{A} = \mbb{R}$. Then we can have, $Q\prs{s,\cdot} = 0$ and $Q\prs{s,\cdot} \sim \mathcal{N}\prs{0,1}$ where the latter is a Gaussian distribution of mean $0$ and variance $1$.
Then, for every state $s$,
\[\mathbb{E}_{a \in \mathcal{A}}\brs{\hat{Q}\prs{s,a}} = 0 = \mathbb{E}_{a \in \mathcal{A}}\brs{Q\prs{s,a}}\]
so $\hat{Q}$ is an unbiased estimator of $Q$, but
\begin{align*}
\mathbb{E}_{\tau \sim p_\theta}\brs{\max_{a'} \hat{Q}\prs{s', a'}} > 0 \text{,}
\mathbb{E}_{\tau \sim p_\theta}\brs{\max_{a'} Q\prs{s', a'}} = \mathbb{E}_{\tau \sim p_\theta}\brs{0} = 0 \text{,}
\end{align*}
as the expected value of the maximum of samples from Gaussian distribution is clearly positive. We get that $\mathcal{B}\hat{Q}$ is not an unbiased estimate for $\mathcal{B}Q$.

\subsection{Tabular Learning}

\subsection{Variance of Q Estimates}

For $N = 1$, the target value
\[y_{i,t} = r_{i,t} + \gamma^N \max_{a_{i, t+1}} Q_{\phi_k}\prs{s_{i,t+1}, a_{i,t+1}}\]
is an approximation for $Q\prs{s_{i,t}, a_{i,t}}$. As we increase $N$, the target value takes more actions which do not maximize the $Q$-value, before eventually maximizing the $Q$-value on the $N$\textsuperscript{th} step.
We expect therefore that as $N$ increases, the model would more closely fit the data, which would result in increased variance. Therefore, the minimal variance would be given when $N=1$ and the maximal one as $N \to \infty$.

\subsection{Function Approximation}

\subsection{Multistep Importance Sampling}

\printbibliography
\printindex

\end{document}
